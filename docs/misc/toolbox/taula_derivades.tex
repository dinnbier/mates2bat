\documentclass{article}
\usepackage[landscape]{geometry}
\usepackage{geometry}
\usepackage[table]{xcolor}% http://ctan.org/pkg/xcolor
\usepackage{fancyhdr}
\usepackage{amsmath, amssymb}
\usepackage{graphicx}
\usepackage{color}
\definecolor{bg}{rgb}{0.89, 0.95, 0.71} % background e3f3b4
\definecolor{ac}{rgb}{0.50, 0.18, 0.41} % accent c491b6
\definecolor{rc}{rgb}{0.77, 0.57, 0.71} % rule 7f2f69
\usepackage{listings}
%catalan packages
\usepackage[catalan]{babel}
\usepackage[T1]{fontenc}
\usepackage[utf8]{inputenc}
%
\lstset{
       basicstyle=\small\ttfamily,
       xleftmargin=1em,
       }
\lstdefinestyle{latex}{language=TeX,
                       backgroundcolor=\color{bg},
                       basicstyle=\small\ttfamily,
                       frame=leftline,
                       xleftmargin=1.4em,
                       framexleftmargin=.8em}
\lstdefinestyle{cmdline}{
                         }
\usepackage{url}
\usepackage[pdftex,
            bookmarks,
            colorlinks=true,
            linkcolor=black,
            plainpages = false,
            pdfpagemode = UseNone,
            pdfstartview = FitH,
            citecolor = ac, urlcolor = ac, filecolor = ac]{hyperref}

\usepackage[para]{footmisc}
% Change horiz room between fn mark and fn hskip from .5em
% Suggested to RF making this settable
\makeatletter
\long\def\@makefntext#1{\leavevmode
\@makefnmark\nobreak
\hskip.05em\relax#1%
}
\makeatother
%\newcommand{\texdoc}[1]{\/\footnote{\protect\texttt{#1}}}
\newcommand{\citelink}[3]{\href{#1}{#2}~\cite{#3}}
\newcommand{\bibliolink}[2]{\href{#1}{\nolinkurl{#1}}, \href{#1}{#2}}

\setlength{\parskip}{0.75ex}
\makeatletter % from emma pease at csli.stanford.edu
% \@startsection {NAME}{LEVEL}{INDENT}{BEFORESKIP}{AFTERSKIP}{STYLE}
%            optional * [ALTHEADING]{HEADING}
%    Generic command to start a section.
%    NAME       : e.g., 'subsection'
%    LEVEL      : a number, denoting depth of section -- e.g., chapter=1,
%                 section = 2, etc.  A section number will be printed if
%                 and only if LEVEL < or = the value of the secnumdepth
%                 counter.
%    INDENT     : Indentation of heading from left margin
%    BEFORESKIP : Absolute value = skip to leave above the heading.
%                 If negative, then paragraph indent of text following
%                 heading is suppressed.
%    AFTERSKIP  : if positive, then skip to leave below heading,
%                       else - skip to leave to right of run-in heading.
%    STYLE      : commands to set style
%  If '*' missing, then increments the counter.  If it is present, then
%  there should be no [ALTHEADING] argument.  A sectioning command
%  is normally defined to \@startsection + its first six arguments.
%
%colors in sections and subsections
%\def\section{\@startsection {section}{1}{\z@}{2.5ex plus .6ex minus
%    .2ex}{1.0ex plus .15ex}{\hspace*{-3em}\Large\bf\color{ac}}}
%\def\subsection{\@startsection{subsection}{2}{\z@}{1.5ex plus .3ex minus
%   .1ex}{.2ex plus .1ex}{\hspace*{-3em}\bf\large\color{ac}}}
\makeatother
\setcounter{secnumdepth}{0}

\newlength{\rulelength}
\setlength{\rulelength}{\linewidth}
\addtolength{\rulelength}{9.5em}
\title{Taula de Derivades}
\author{INS Vilafant 17/18}
\date{\today}

\pagestyle{plain}
\begin{document}
\thispagestyle{empty}
%\maketitle
\setlength{\unitlength}{1in}

\makeatletter
\vspace*{3ex}
\par\noindent{\hspace*{-1.5em}\LARGE\bf \@title}
\vspace*{-1.2ex}
\par\noindent{{\hspace*{-1.5em}\rule{\rulelength}{1.05pt}}}
\vspace*{-.5ex}
\par\noindent{\hspace*{-1.5em}\large \@author}
\vspace*{5ex}
\makeatother







\begin{center}
	\begin{tabular}{ | l |l|}
		\hline
		\cellcolor{gray}\textbf{Funció} & \cellcolor{gray}\textbf{Derivada}  \\
		\hline
		Constant &$(k)^{\prime}=0 \forall k \in \mathbb{R}$\\
		\hline
	Identitat      &$x^{\prime}=1 $\\
		\hline
	 Suma      &$(f(x)+g(x))^{\prime}=f^{\prime}(x)+g^{\prime}(x)$\\
	 	\hline
	 Producte per un nombre      &$(k\cdot f(x))^{\prime}=k \cdot f^{\prime}(x)$\\
	 	\hline
	 Potència      &$(x^n)^{\prime}=n\cdot x^{n-1}$\\
	\hline
	Polinomi:  &$[a_n x^n+a_{n-1}x^{n-1}+...+a_2 x^2+a_1 x+a_0]^\prime=n\cdot a_n x^{n-1}+(n-1)\cdot a_{n-1}x^{n-2}+...+2\cdot a_2 x+a_1$\\
		\hline
	 Producte      &$(f(x)\cdot g(x))^{\prime}=f^{\prime}(x)\cdot g(x)+f(x)\cdot g^{\prime} (x)$\\
	 	\hline
	 Quocient      &$\Big(\frac{f(x)}{g(x)}\Big)^{\prime}=\frac{f^{\prime}(x)\cdot g(x)-f(x)\cdot g^{\prime}(x)}{g(x)^2}$\\
	 	\hline
	 Trigonomètriques      &$\big(\sin(x)\big)^{\prime}=\cos(x)$\\
	       &$\big(\cos(x)\big)^{\prime}=-\sin(x)$\\
	      &$\big(\tan(x)\big)^{\prime}=1+\tan^2(x)$\\
	 	\hline
	 Inverses Trigonomètriques      &$\big(arcsin(x)\big)^{\prime}=\frac{1}{\sqrt{1-x^2}}$\\
	       &$\big(arccos(x)\big)^{\prime}=-\frac{1}{\sqrt{1-x^2}}$\\
	      &$\big(arctan(x)\big)^{\prime}=\frac{1}{1+x^2}$\\
		\hline
	 Composició (regla de la cadena)&$\big(f(g(x))\big)^{\prime}=f^{\prime}\big(g(x)\big)\cdot g^{\prime}(x)$\\
		\hline
	Exponencials&$\big(e^x\big)^{\prime}=e^x$\\
	&$\big(a^x\big)^{\prime}=a^x\cdot ln(a)$\\
		\hline
	Logarítmiques& $\big(ln(x)\big)^{\prime}=\frac{1}{x}$\\
	&$\big(log_a(x)\big)^{\prime}=\frac{1}{x}\cdot \frac{1}{ln(a)}$\\
	
		\hline

	\end{tabular}
\end{center}



\end{document}
